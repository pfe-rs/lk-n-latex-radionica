% Velicina i tip stranice
\documentclass[a4paper]{article}

%% Dodatak za margine
\usepackage[a4paper,top=2cm,bottom=2cm,left=2cm,right=2cm,marginparwidth=1.75cm]{geometry}

%% Dodaci za jezik
\usepackage{hyphsubst}
\usepackage{ucs}
\usepackage[T1]{fontenc}
\usepackage[utf8x]{inputenc}
\usepackage[serbian]{babel}

%% Korisni dodaci
\usepackage{amsfonts}
\usepackage{amsmath}
\usepackage{amssymb}
\usepackage{booktabs}
\usepackage{enumerate}
\usepackage{enumitem}
\usepackage{esint}
\usepackage{gensymb}
\usepackage{graphicx}
\usepackage[hidelinks]{hyperref}
\usepackage{listings}
\usepackage{lipsum}
\usepackage{setspace}
\usepackage{lmodern}
\usepackage{mathrsfs}
\usepackage{mathtools}
\usepackage{minted}
\usepackage{multicol}
\usepackage{multicol}
\usepackage{csquotes}
\usepackage{physics}

% Označavanje figura
\usepackage{caption}
\captionsetup{justification=justified,
   format=plain,font=small,labelfont=sc,margin=50pt}

% Formatiranje
\setstretch{1.2}
\setlength{\parskip}{1em}
\setlength{\parindent}{0em}
\setlength{\abovedisplayskip}{0pt}
\setlength{\belowdisplayskip}{0pt}

% Uputstvo:
% U svoj TeX dokument na vrh dodajte
% % Velicina i tip stranice
\documentclass[a4paper]{article}

%% Dodatak za margine
\usepackage[a4paper,top=2cm,bottom=2cm,left=2cm,right=2cm,marginparwidth=1.75cm]{geometry}

%% Dodaci za jezik
\usepackage{hyphsubst}
\usepackage{ucs}
\usepackage[T1]{fontenc}
\usepackage[utf8x]{inputenc}
\usepackage[serbian]{babel}

%% Korisni dodaci
\usepackage{amsfonts}
\usepackage{amsmath}
\usepackage{amssymb}
\usepackage{booktabs}
\usepackage{enumerate}
\usepackage{enumitem}
\usepackage{esint}
\usepackage{gensymb}
\usepackage{graphicx}
\usepackage[hidelinks]{hyperref}
\usepackage{listings}
\usepackage{lipsum}
\usepackage{setspace}
\usepackage{lmodern}
\usepackage{mathrsfs}
\usepackage{mathtools}
\usepackage{minted}
\usepackage{multicol}
\usepackage{multicol}
\usepackage{csquotes}
\usepackage{physics}

% Označavanje figura
\usepackage{caption}
\captionsetup{justification=justified,
   format=plain,font=small,labelfont=sc,margin=50pt}

% Formatiranje
\setstretch{1.2}
\setlength{\parskip}{1em}
\setlength{\parindent}{0em}
\setlength{\abovedisplayskip}{0pt}
\setlength{\belowdisplayskip}{0pt}

% Uputstvo:
% U svoj TeX dokument na vrh dodajte
% % Velicina i tip stranice
\documentclass[a4paper]{article}

%% Dodatak za margine
\usepackage[a4paper,top=2cm,bottom=2cm,left=2cm,right=2cm,marginparwidth=1.75cm]{geometry}

%% Dodaci za jezik
\usepackage{hyphsubst}
\usepackage{ucs}
\usepackage[T1]{fontenc}
\usepackage[utf8x]{inputenc}
\usepackage[serbian]{babel}

%% Korisni dodaci
\usepackage{amsfonts}
\usepackage{amsmath}
\usepackage{amssymb}
\usepackage{booktabs}
\usepackage{enumerate}
\usepackage{enumitem}
\usepackage{esint}
\usepackage{gensymb}
\usepackage{graphicx}
\usepackage[hidelinks]{hyperref}
\usepackage{listings}
\usepackage{lipsum}
\usepackage{setspace}
\usepackage{lmodern}
\usepackage{mathrsfs}
\usepackage{mathtools}
\usepackage{minted}
\usepackage{multicol}
\usepackage{multicol}
\usepackage{csquotes}
\usepackage{physics}

% Označavanje figura
\usepackage{caption}
\captionsetup{justification=justified,
   format=plain,font=small,labelfont=sc,margin=50pt}

% Formatiranje
\setstretch{1.2}
\setlength{\parskip}{1em}
\setlength{\parindent}{0em}
\setlength{\abovedisplayskip}{0pt}
\setlength{\belowdisplayskip}{0pt}

% Uputstvo:
% U svoj TeX dokument na vrh dodajte
% \input{./dataset/izvestaj.tex}
% kako biste preuzeli podešavanja za izveštaj,
% slično fajlu smernice.tex

% kako biste preuzeli podešavanja za izveštaj,
% slično fajlu smernice.tex

% kako biste preuzeli podešavanja za izveštaj,
% slično fajlu smernice.tex


\title{Interaktivni uvod u \LaTeX}
\author{Filip Parag}

\begin{document}

\maketitle

\section*{Sadržaj}

\begin{enumerate}
   \item Dokument
   \item Naslovi
   \item Teskt i pasusi
   \item Matematika
   \item Figure
   \item Tabele
   \item Liste
   \item Kod
\end{enumerate}

\pagebreak

\section*{Dokument}

\begin{minted}{latex}
   \title{Interaktivni uvod u \LaTeX}
   \author{
      Filip Parag\\Naučno-inženjerski centar \enquote{PFE}
      \and
      Petar Petrović\\Član naše mašte
   }

   \maketitle
   \tableofcontents
\end{minted}

\pagebreak

\section*{Naslovi}

\begin{minted}{latex}
  \section[Skraćeni naziv]{Odeljak sa indeksom}
  \subsection{Pododeljak sa indeksom}
  \subsubsection{Pododeljak sa indeksom}

  \section*{Odeljak bez indeksa} \label{labela_naslova}
\end{minted}

\pagebreak

\section*{Tekst i pasusi}

\begin{minted}{latex}
   \lipsum[1]
   \lipsum[1]

   \lipsum[1] \\
   \lipsum[1]

   \footnote{Fusnota}
   \marginpar{Tekst unutar margine}
\end{minted}

\begin{minted}{latex}
   \textbf{Podebljan tekst}
   \textit{Italiciziran tekst}
   \texttt{Blokovski tekst}
   \underline{Podvučen tekst}
   \enquote{Navođen tekst}
   \(Matematički izraz\)
   \ref{labela}
\end{minted}

\pagebreak

\section*{Matematika}

\begin{minted}{latex}
   % Jednačina
   \begin{equation}
      \nabla \cross \mathbf{E} = - \frac{1}{c} \frac{\partial\mathbf{B}}{\partial t}
      \label{eq:faradej}
   \end{equation}

   % Sistem jednačina
   \begin{alignat*}{5}
      \dot{x_1} & ={} & 2 x_1 & {} & {} & {}-{} & 2 x_3 \\
      \dot{x_2} & ={} & {} & {} & x_2 & {}+{} & 5 x_3 \\
      \dot{x_3} & ={} & x_1 & {}+{} & x_2 & {}+{} & x_3
    \end{alignat*}

   % Matrica
   \[
      A = \left[ \begin{matrix}
         2  &  0 & -2 \\
         0  &  1 &  5 \\
         1  &  1 &  1
      \end{matrix} \right]
   \]
\end{minted}

\pagebreak

\section*{Figure}

\begin{minted}{latex}
   % h - postavi otprilike ovde
   % t - postavi na vrh stranice
   % b - postavi na dno stranice
   % p - postavi na zasebnu stranicu za figure
   \begin{figure}[h]
      \includegraphics[height=8cm,width=5cm,keepaspectratio]{./slika.png}
      \centering
      \caption{Matematičko klatno}
      \label{im:idealnoklatno}
   \end{figure}
\end{minted}

\pagebreak

\section*{Tabele}

\begin{minted}{latex}
   \begin{table}[]
      \centering
      % l - Poravnavanje levo
      % c - Poravnavanje u sredini
      % r - Poravnavanje desno
      % | - vertikalna podela
      % \hline - horizontalna podela
      \begin{tabular}{@{}cc@{}}
      \toprule
         Godina & Prinosi \\ \midrule
         2019   & 28.9\%  \\ \hline
         2020   & 16.3\%  \\
         2021   & 26.9\%  \\
         2022   & -19.4\% \\
         2023   & 14.7\%  \\ \bottomrule
      \end{tabular}
      \caption{Godišnji prinosi indeksa S\&P 500}
      \label{tab:sp500}
   \end{table}
\end{minted}

\pagebreak

\section*{Liste}

\begin{minted}{latex}
   \begin{itemize}
      \item Lemilica
      \item Multimetar
      \item Osciloskop
   \end{itemize}

   \begin{enumerate}[label=\alph*)]
      \item Doručak
      \item Ručak
      \item Večera
   \end{enumerate}
\end{minted}

\pagebreak

\section*{Kod}

\begin{listing}[h]
   \begin{minted}{cpp}
      #define DELAY 500
      void loop() {
         digitalWrite(2, HIGH);
         delay(DELAY);
         digitalWrite(2, LOW);
         delay(DELAY);
      }
   \end{minted}
   \label{ls:arduino}
   \caption{Primer Arduino koda za kontrolu LED}
\end{listing}


\end{document}
